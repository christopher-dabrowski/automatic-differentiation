%% LaTeX template based on the https://mirrors.mit.edu/CTAN/macros/latex/contrib/IEEEtran/bare_conf.tex

\documentclass[conference,a4paper]{IEEEtran}
\input{config}

\begin{document}
\title{State of Automatic Differentiation\\ Literature Review}

\author{
\IEEEauthorblockN{Krzysztof Dąbrowski}
\IEEEauthorblockA{Faculty of Electrical Engineering\\
Warsaw University of Technology\\
Warsaw 00-662, Poland\\
Email: krzysztof.dabrowski7.stud@pw.edu.pl
}}

\maketitle

% Mój dokument ma zająć od połowy do całej strony

\section{Introduction and Research Context}
% omówienie kontekstu badawczego i znaczenia tematu w literaturze

In my research I'll be studding fast implementation of automatic differentiation in The Julia programming language \cite{JuliaLanguage:Homepage}.
The topic of automatic differentiation is very important in the fields of machine learning and optimization.
The main algorithms used for automatic differentiation are well research in the popular prograding language like C, but there are not as well studied in Julia.
As Julia is focused on high performance and ease of use, especially in academic fields, I see a lof of value in expiring how the automatic differentiation can be implemented in Julia in a performance oriented way.


\section{Current Research}
% podsumowanie najważniejszych badań (+ ew. wskazanie braków, rozbieżności)
% Przejście w tył

\section{Research Justification}
% uzasadnienie potrzeby przeprowadzenia własnego badania, wskazanie, w jaki sposób wpisuje się ono w istniejącą literaturę

\section{Literature Impact}
% wnioski wynikające z analizy literaturowej i ich implikacje dla prowadzonych badań

\bibliographystyle{IEEEtran}
\bibliography{Bibliography}

\end{document}

