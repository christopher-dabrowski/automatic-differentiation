%% LaTeX template based on the https://mirrors.mit.edu/CTAN/macros/latex/contrib/IEEEtran/bare_conf.tex

\documentclass[conference,a4paper]{IEEEtran}
\input{config}

\begin{document}
\title{State of Automatic Differentiation\\ Literature Review}

\author{
\IEEEauthorblockN{Krzysztof Dąbrowski}
\IEEEauthorblockA{Faculty of Electrical Engineering\\
Warsaw University of Technology\\
Warsaw 00-662, Poland\\
Email: krzysztof.dabrowski7.stud@pw.edu.pl
}}

\maketitle

% Mój dokument ma zająć od połowy do całej strony

\section{Introduction and Research Context}
% omówienie kontekstu badawczego i znaczenia tematu w literaturze

I wish you the best of success. \cite{IEEEexample:article_typical}

\section{Current Research}
% podsumowanie najważniejszych badań (+ ew. wskazanie braków, rozbieżności)


\section{Research Justification}
% uzasadnienie potrzeby przeprowadzenia własnego badania, wskazanie, w jaki sposób wpisuje się ono w istniejącą literaturę

\section{Literature Impact}
% wnioski wynikające z analizy literaturowej i ich implikacje dla prowadzonych badań

\bibliographystyle{IEEEtran}
\bibliography{Bibliography}

\end{document}

